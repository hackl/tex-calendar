%=============================================================================
% File:  week.tex -- Calender showing the week
% Author(s): Juergen Hackl <hackl.j@gmx.at>
% Creation:  27 Jul 2017
% Time-stamp: <Sam 2017-07-29 13:15 juergen>
%
% Copyright (c) 2017 Juergen Hackl <hackl.j@gmx.at>
%
% More information on LaTeX: http://www.latex-project.org/
%=============================================================================

%%%%%%%%%%%%%%%%%%%%%%%%%%%%%%%%%%%%%%%%%%%%%%%%%%%%%%%%%%%%%%%%%%%%%%
\documentclass[a4paper,landscape]{article}

%\usepackage[cm]{fullpage}
\usepackage[margin=0pt]{geometry}
 % \geometry{
 % a4paper,
 % total={297mm,210mm},
 % landscape,
 % }

\usepackage{ifthen}

\def\stripzero#1{\expandafter\stripzerohelp#1}
\def\stripzerohelp#1{\ifx 0#1\expandafter\stripzerohelp\else#1\fi}

\usepackage{tikz}
\usetikzlibrary{calendar,shadows,patterns}
\usetikzlibrary{positioning,calc}
\usetikzlibrary{backgrounds}
% Names of Holidays are inserted by employing this macro
\def\termin#1#2{
  \node [anchor=north west, text width= 3.4cm] at
    ($(cal-#1.north west)+(3em, 0em)$) {\tiny{#2}};
}

\usepackage{fontspec}
\setmainfont[ItalicFont={Fira Sans Light Italic},%
BoldFont={Fira Sans},%
BoldItalicFont={Fira Sans Italic}]%
{Fira Sans Light}


% \def\termin#1#2{
% \node [text width= 3.4cm] at
% ($(cal-#1)+(0em, 0em)$) {\textcolor{red\tiny{#2}};
%  }


%%%%%%%%%%%%%%%%%%%%%%%%%%%%%%%%%%%%%%%%%%%%%%%%%%%%%%%%%%%%%%%%%%%%%%
%% DOCUMENT
\begin{document}

\tikzstyle{every shadow}=[opacity=.4,shadow xshift=2pt,shadow yshift=-2pt,fill=black!60]
\tikzstyle{caption}=[drop shadow,rounded corners=5pt,inner sep=4pt,draw=black!50,fill=black!5]
\tikzstyle{plot legend}=[
   rounded corners=2.5pt,inner xsep=3pt,inner ysep=2pt,
   draw=black!50,fill=white,
   font=\footnotesize,cells={anchor=center},
   nodes={inner sep=2pt,text depth=0.15em,rounded corners=0pt,right}]

\begin{tikzpicture}[remember picture,overlay,shift={(current page.north west)}]%[remember picture,overlay,xshift=-0cm]

%\newcommand{\weekbegin}{2017-06-19}
\newcommand{\weekend}{2017-06-25}
% \newcommand{\weekbegin}{2017-07-03}
% \newcommand{\weekend}{2017-07-09}


% \newcommand{\weekbegin}{2017-01-02}
% \newcommand{\weekend}{2017-01-08}

% \newcommand{\weekbegin}{2017-12-18}
% \newcommand{\weekend}{2017-12-24}

% \newcommand{\weekbegin}{2017-08-14}
% \newcommand{\weekend}{2017-08-20}


\newlength{\daywidth}
\newlength{\dayheight}
\newlength{\weekheight}
\newlength{\pagespace}
\newlength{\boxspace}
\newlength{\bugfixspace}
\newcommand{\notesoffset}{0}

\setlength{\daywidth}{34mm}
\setlength{\dayheight}{25mm}
\newcommand{\dayh}{25}
\setlength{\weekheight}{5mm}
\newcommand{\weekh}{5}
\setlength{\pagespace}{12mm}
\setlength{\boxspace}{2mm}
\setlength{\bugfixspace}{0mm}
%\setlength{\notesoffset}{0mm}
\newcommand{\notesheight}{18.5}

\newlength{\defaultpgflinewidth}
\setlength{\defaultpgflinewidth}{.5pt}%{\pgflinewidth}

\definecolor{linecolor}{gray}{.22}
\definecolor{headingcolor}{gray}{.22}
\definecolor{weekcolor}{gray}{0.35}
\definecolor{weekendcolor}{gray}{0.8}
\definecolor{monthcolor}{gray}{0.6}
\definecolor{htextcolor}{gray}{1}
\definecolor{textcolor}{gray}{.22}
\definecolor{termincolor}{RGB}{58,110,143}
\definecolor{sundaycolor}{gray}{.22}
\definecolor{indexcolor}{gray}{.22}
 
\tikzset{myline/.style={draw, linecolor, line width = \defaultpgflinewidth}}
\tikzset{mythinline/.style={draw, linecolor!50, line width = 0.3pt}}


\begin{scope}[shift={(40.5mm,-13mm)}]
\calendar(cal)
[dates=\weekbegin to \weekbegin,
execute before day scope={%
      month caption:
      \ifdate{equals=\weekbegin}{%
        %
        \newcount\myBcount
        \newcount\myEcount
%
        \pgfcalendardatetojulian{\weekbegin}{\myBcount}
        \pgfcalendardatetojulian{\weekend}{\myEcount}
        \pgfcalendarjuliantodate{\the\myBcount}{\myByear}{\myBmonth}{\myBday}
        \pgfcalendarjuliantodate{\the\myEcount}{\myEyear}{\myEmonth}{\myEday}
%
        \setlength{\bugfixspace}{0mm}
      \ifdate{Thursday}{\setlength{\bugfixspace}{\pagespace}}{}%
      \ifdate{Friday}{\setlength{\bugfixspace}{\pagespace}}{}%
      \ifdate{Saturday}{\setlength{\bugfixspace}{\pagespace}}{}%
      \ifdate{Sunday}{\setlength{\bugfixspace}{\pagespace}}{}%
%         \node[yshift=\weekheight, xshift=-\daywidth-\bugfixspace,
%           minimum width=2*\daywidth,
%           text width=2*\daywidth,
%           %draw,
%           % text width=4cm,
%           every month,
%           font=\Large\bf,
%           align=left,
% %               caption,
% %          above right =0mm,
%           anchor=north west,
%                inner sep=0em]
%             {\MakeUppercase{\pgfcalendarmonthshortname{\myBmonth}\myBday ~- \pgfcalendarmonthshortname{\myEmonth}\myEday}};
       \ifnum \myBmonth > 7
       \draw [loosely dotted,black!50] ($(current page.north east)-(.6,\stripzero{\myBmonth}*1.75-14)$) --++ (0,-21);
        \node at ($(current page.north east)-(0,\stripzero{\myBmonth}*1.75-14)$) [
        fill=indexcolor,
                  minimum width=17.5mm,
                  minimum height=6mm,
                  anchor = south east,
                  rotate = 90,
                  text = htextcolor,
        ]{\MakeUppercase{\pgfcalendarmonthshortname{\myBmonth} \myBmonth}};
        \else
       \draw[loosely dotted,black!50] ($(current page.north east)-(.6,\stripzero{\myBmonth}*1.75+7)$) --++ (0,-21);
        \node at ($(current page.north east)-(0,\stripzero{\myBmonth}*1.75+7)$) [
        fill=indexcolor,
                  minimum width=17.5mm,
                  minimum height=6mm,
                  anchor = south east,
                  rotate = 90,
                  text = htextcolor,
        ]{\MakeUppercase{\pgfcalendarmonthshortname{\myBmonth} \myBmonth}};
        \fi
          }{}
      % week day captions:
      % \ifdate{at most=\weekbegin+6}{
      %   \path (\pgfcalendarcurrentweekday\daywidth,0)
      %     node[xshift=(\daywidth-\boxspace/2)/2+.3pt/2,
      %     node distance=-\defaultpgflinewidth,
      %          font=\normalsize,
      %          above,
      %          anchor=south,
      %          mythinline,
      %          minimum width=\daywidth-\boxspace/2,
      %          text width=\daywidth-2*\boxspace,
      %          minimum height=\weekheight,
      %          inner sep = 0.0pt,
      %          outer sep = 0pt,
      %          text=htextcolor,
      %          week bg,
      %          align=left,
      %          label={right,xshift=-7mm:\color{htextcolor}\bf\pgfcalendarcurrentday}
      %          ] {%
      %            \MakeUppercase{\pgfcalendarweekdayname{\pgfcalendarcurrentweekday}}%
      %     };
      % }{}
    },
    % execute at begin day scope={%
    %   \pgftransformxshift{\pgfcalendarcurrentweekday\daywidth}%
    % },
    % execute after day scope={%
    %   \ifdate{Wednesday}{%
    %     \pgftransformxshift{\pagespace}
    %   }{}%
    % },
    % execute after day scope={%
    %   \ifdate{Sunday}{%
    %     \pgftransformyshift{-\dayheight}%
    %     \pgftransformxshift{-\pagespace}
    %   }{}%
    % },
%    week list,
%    day xshift=\daywidth,
%    day yshift=\dayheight,
%     day bg/.style=,
%     every day text/.style={font=\footnotesize\bf},
%     week bg/.style={fill=weekcolor},
%     day code={%
%       \node[anchor=north west,
%             mythinline,
%             minimum width=\daywidth-\boxspace/2,
%             minimum height=\dayheight,
%             day bg]{};%
%       \node[anchor=north west,
%             every day text,
%             inner sep=7pt]{};%
%     },
%   every month/.append style={anchor=base},
% %    month label above centered
  ]
  % if (Sunday) [sundaycolor]
  % if (weekend) [day bg/.append style={fill=weekendcolor}]
%  if (equals=2017-06-20)
  % if (between=2017-06-20 and 2017-06-22)
  % {\node at (0,0) [anchor=north west,
  %   inner sep = 0pt,
  %   outer sep = 0pt,
  %   minimum width=\daywidth-\boxspace/2,
  %   text width=\daywidth-\boxspace/2,
  %   minimum height=\dayheight,
  %   fill=termincolor,opacity=.5,align=center,label={below,yshift=5mm:Urlaub}]{};}
  %if (equals=2017-06-20)[day bg/.append style={fill=red!20,on background layer}]
  % if (at most=\YYMM-01+-1,at least=\YYMM-last+1)
  %   [%day bg/.append style={pattern=north east lines,pattern color=black!50},
  %    day bg/.append style={fill=monthcolor},
  %    every day text/.append style={font=\footnotesize}]
;

\tikzset{heading/.style={myline, node distance=-\defaultpgflinewidth+\boxspace, minimum width=\daywidth-\boxspace,minimum height=\weekheight,inner sep = 0pt,outer sep = 0pt,text=htextcolor,fill=headingcolor, font=\normalsize,}}
\tikzset{headingbox/.style={myline, node distance=-\defaultpgflinewidth, minimum width=\daywidth-\boxspace,inner sep = 0pt,outer sep = 0pt,}}


\begin{scope}[yshift=0 mm ,xshift=-\daywidth]
          \node (x) at (0,0) [heading,minimum width=4*\daywidth,anchor = south west]{\MakeUppercase{Notes}};
          \node [right=\pagespace of x,heading,minimum width=4*\daywidth] (y) {\MakeUppercase{Notes}};
          %
\foreach \z in {0,.5,...,\notesheight} {
\draw [myline]
let 
\p1=(x.south west),
\p2=(x.south east)
in
($(\x1,-\z)$) --++ (4*\daywidth,0);
}
%
\foreach \z in {0,.5,...,\notesheight} {
\draw [myline]
let 
\p1=(y.south west),
\p2=(y.south east)
in
($(\x1,-\z)$) --++ (4*\daywidth,0);
}
%
%
        \end{scope}
















\end{scope}




% \draw (0,0) -- (0.6,-.8);
% \draw (0,-21) --++ (0.6,.8);
% \termin{2017-08-02}{Pfingstsonntag}
\end{tikzpicture}

\begin{tikzpicture}[remember picture,overlay]% anchor = west]% xshift=-0cm]
\tikzset{mymarkline/.style={draw, black!50, line width = 0.6pt}}
\tikzset{mycircle/.style={black!50}}

\newlength{\marklength}
\setlength{\marklength}{1.5mm}

\newlength{\diam}
\setlength{\diam}{1pt}

\fill [mycircle] ($(current page.north)-(0,.8)$) circle (\diam);
\fill [mycircle] ($(current page.north)-(0,3.4)$) circle (\diam);
\fill [mycircle] ($(current page.north)-(0,6.4)$) circle (\diam);
\fill [mycircle] ($(current page.north)-(0,9)$) circle (\diam);
\fill [mycircle] ($(current page.south)+(0,.8)$) circle (\diam);
\fill [mycircle] ($(current page.south)+(0,3.4)$) circle (\diam);
\fill [mycircle] ($(current page.south)+(0,6.4)$) circle (\diam);
\fill [mycircle] ($(current page.south)+(0,9)$) circle (\diam);


\draw [mymarkline] (current page.north west) --++ (\marklength,0);
\draw [mymarkline] (current page.north west) --++ (0,-\marklength);

\draw [mymarkline] (current page.north east) --++ (-\marklength,0);
\draw [mymarkline] (current page.north east) --++ (0,-\marklength);


\draw [mymarkline] (current page.south west) --++ (\marklength,0);
\draw [mymarkline] (current page.south west) --++ (0,\marklength);

\draw [mymarkline] (current page.south east) --++ (-\marklength,0);
\draw [mymarkline] (current page.south east) --++ (0,\marklength);


\end{tikzpicture}


%%%%%%%%%%%%%%%%%%%%%%%%%%%%%%%%%%%%%%%%%%%%%%%%%%%%%%%%%%%%%%%%%%%%%%

\end{document}

%%% Local Variables:
%%% mode: latex
%%% TeX-master: t
%%% End:
